\documentclass[]{article}
\usepackage[T1]{fontenc}
\usepackage{lmodern}
\usepackage{amssymb,amsmath}
\usepackage{ifxetex,ifluatex}
\usepackage{fixltx2e} % provides \textsubscript
% use upquote if available, for straight quotes in verbatim environments
\IfFileExists{upquote.sty}{\usepackage{upquote}}{}
\ifnum 0\ifxetex 1\fi\ifluatex 1\fi=0 % if pdftex
  \usepackage[utf8]{inputenc}
\else % if luatex or xelatex
  \ifxetex
    \usepackage{mathspec}
    \usepackage{xltxtra,xunicode}
  \else
    \usepackage{fontspec}
  \fi
  \defaultfontfeatures{Mapping=tex-text,Scale=MatchLowercase}
  \newcommand{\euro}{€}
\fi
% use microtype if available
\IfFileExists{microtype.sty}{\usepackage{microtype}}{}
\usepackage{color}
\usepackage{fancyvrb}
\newcommand{\VerbBar}{|}
\newcommand{\VERB}{\Verb[commandchars=\\\{\}]}
\DefineVerbatimEnvironment{Highlighting}{Verbatim}{commandchars=\\\{\}}
% Add ',fontsize=\small' for more characters per line
\newenvironment{Shaded}{}{}
\newcommand{\KeywordTok}[1]{\textcolor[rgb]{0.00,0.44,0.13}{\textbf{{#1}}}}
\newcommand{\DataTypeTok}[1]{\textcolor[rgb]{0.56,0.13,0.00}{{#1}}}
\newcommand{\DecValTok}[1]{\textcolor[rgb]{0.25,0.63,0.44}{{#1}}}
\newcommand{\BaseNTok}[1]{\textcolor[rgb]{0.25,0.63,0.44}{{#1}}}
\newcommand{\FloatTok}[1]{\textcolor[rgb]{0.25,0.63,0.44}{{#1}}}
\newcommand{\CharTok}[1]{\textcolor[rgb]{0.25,0.44,0.63}{{#1}}}
\newcommand{\StringTok}[1]{\textcolor[rgb]{0.25,0.44,0.63}{{#1}}}
\newcommand{\CommentTok}[1]{\textcolor[rgb]{0.38,0.63,0.69}{\textit{{#1}}}}
\newcommand{\OtherTok}[1]{\textcolor[rgb]{0.00,0.44,0.13}{{#1}}}
\newcommand{\AlertTok}[1]{\textcolor[rgb]{1.00,0.00,0.00}{\textbf{{#1}}}}
\newcommand{\FunctionTok}[1]{\textcolor[rgb]{0.02,0.16,0.49}{{#1}}}
\newcommand{\RegionMarkerTok}[1]{{#1}}
\newcommand{\ErrorTok}[1]{\textcolor[rgb]{1.00,0.00,0.00}{\textbf{{#1}}}}
\newcommand{\NormalTok}[1]{{#1}}
\ifxetex
  \usepackage[setpagesize=false, % page size defined by xetex
              unicode=false, % unicode breaks when used with xetex
              xetex]{hyperref}
\else
  \usepackage[unicode=true]{hyperref}
\fi
\hypersetup{breaklinks=true,
            bookmarks=true,
            pdfauthor={},
            pdftitle={},
            colorlinks=true,
            citecolor=blue,
            urlcolor=blue,
            linkcolor=magenta,
            pdfborder={0 0 0}}
\urlstyle{same}  % don't use monospace font for urls
\setlength{\parindent}{0pt}
\setlength{\parskip}{6pt plus 2pt minus 1pt}
\setlength{\emergencystretch}{3em}  % prevent overfull lines
\setcounter{secnumdepth}{0}

\author{}
\date{}

\begin{document}

\section{rChartsでレジェンドプレーヤーの成績を可視化したい}\label{rchartsux3067ux30ecux30b8ux30a7ux30f3ux30c9ux30d7ux30ecux30fcux30e4ux30fcux306eux6210ux7e3eux3092ux53efux8996ux5316ux3057ux305fux3044}

Lahmanのデータベースを使って, 歴代の打者成績推移をrChartsで可視化します.

データは http://seanlahman.com/files/database/lahman-csv\_2014-02-14.zip
でダウンロードできます.

解凍したら, Batting.csvとMaster.csvを使います.

\begin{Shaded}
\begin{Highlighting}[]
\KeywordTok{library}\NormalTok{(rCharts)}
\KeywordTok{library}\NormalTok{(RPostgreSQL)}
\KeywordTok{library}\NormalTok{(dplyr)}
\KeywordTok{library}\NormalTok{(magrittr)}
\KeywordTok{library}\NormalTok{(xtable)}
\CommentTok{# 各打者のシーズン記録のまとめデータ: Batting.csv}
\NormalTok{dat =}\StringTok{ }\KeywordTok{fread}\NormalTok{(}\StringTok{"Batting.csv"}\NormalTok{)}
\end{Highlighting}
\end{Shaded}

データの中身はこんな感じです.

\begin{Shaded}
\begin{Highlighting}[]
\KeywordTok{head}\NormalTok{(dat) %>%}\StringTok{ }\KeywordTok{xtable}\NormalTok{() %>%}\StringTok{ }\KeywordTok{print}\NormalTok{(}\DataTypeTok{type=}\StringTok{"html"}\NormalTok{)}
\end{Highlighting}
\end{Shaded}

playerID

yearID

stint

teamID

lgID

G

G\_batting

AB

R

H

X2B

X3B

HR

RBI

SB

CS

BB

SO

IBB

HBP

SH

SF

GIDP

G\_old

1

aardsda01

2004

1

SFN

NL

11

11

0

0

0

0

0

0

0

0

0

0

0

0

0

0

0

0

11

2

aardsda01

2006

1

CHN

NL

45

43

2

0

0

0

0

0

0

0

0

0

0

0

0

1

0

0

45

3

aardsda01

2007

1

CHA

AL

25

2

0

0

0

0

0

0

0

0

0

0

0

0

0

0

0

0

2

4

aardsda01

2008

1

BOS

AL

47

5

1

0

0

0

0

0

0

0

0

0

1

0

0

0

0

0

5

5

aardsda01

2009

1

SEA

AL

73

3

0

0

0

0

0

0

0

0

0

0

0

0

0

0

0

0

6

aardsda01

2010

1

SEA

AL

53

4

0

0

0

0

0

0

0

0

0

0

0

0

0

0

0

0

\begin{Shaded}
\begin{Highlighting}[]

\CommentTok{# 選手をidで扱うのは淋しいので, フルネームを調べます}
\CommentTok{# フルネームはMaster.csvから作ります.}
\NormalTok{fullname_and_id =}\StringTok{ }
\StringTok{  }\KeywordTok{fread}\NormalTok{(}\StringTok{"Master.csv"}\NormalTok{) %>%}\StringTok{ }
\StringTok{  }\KeywordTok{mutate}\NormalTok{(}\DataTypeTok{fullname =} \KeywordTok{paste}\NormalTok{(nameFirst, nameLast, }\DataTypeTok{sep=}\StringTok{" "}\NormalTok{)) %>%}\StringTok{ }
\StringTok{  }\NormalTok{dplyr::}\KeywordTok{select}\NormalTok{(lahman40ID, fullname) %>%}\StringTok{ }
\StringTok{  }\KeywordTok{setnames}\NormalTok{(}\KeywordTok{c}\NormalTok{(}\StringTok{"playerID"}\NormalTok{, }\StringTok{"fullname"}\NormalTok{))}

\CommentTok{# 2012年までに3000本安打を超えている選手で遊びます}
\NormalTok{bat_over_3000 =}\StringTok{ }
\StringTok{  }\NormalTok{dat %>%}\StringTok{ }\KeywordTok{group_by}\NormalTok{(playerID) %>%}\StringTok{ }
\StringTok{  }\NormalTok{dplyr::}\KeywordTok{summarise}\NormalTok{(}\DataTypeTok{HIT =} \KeywordTok{sum}\NormalTok{(H)) %>%}\StringTok{ }
\StringTok{  }\KeywordTok{filter}\NormalTok{(HIT >=}\StringTok{ }\DecValTok{3000}\NormalTok{) %>%}\StringTok{ }
\StringTok{  }\KeywordTok{select}\NormalTok{(playerID)}

\CommentTok{# フルネームとIDを統合}
\NormalTok{batters =}\StringTok{ }\NormalTok{bat_over_3000 %>%}\StringTok{ }\KeywordTok{inner_join}\NormalTok{(fullname_and_id, }\DataTypeTok{by=}\StringTok{"playerID"}\NormalTok{)}

\CommentTok{# 3000本打った選手の成績まとめ}
\NormalTok{batters_data =}\StringTok{ }\NormalTok{batters %>%}\StringTok{ }\KeywordTok{inner_join}\NormalTok{(dat, }\DataTypeTok{by =} \StringTok{"playerID"}\NormalTok{)}

\CommentTok{# シーズン中に移籍すると, データが別の行になります. }
\CommentTok{# 欲しいのは年度別の成績なので, 年度が同じなら成績はマージします}
\CommentTok{# 移籍はなかったことにします.}
\NormalTok{batters_data_hit =}\StringTok{ }
\StringTok{  }\NormalTok{batters_data %>%}
\StringTok{  }\NormalTok{dplyr::}\KeywordTok{select}\NormalTok{(fullname, yearID, G, AB, H, HR, SO) %>%}\StringTok{ }
\StringTok{  }\KeywordTok{group_by}\NormalTok{(fullname, yearID) %>%}
\StringTok{  }\NormalTok{dplyr::}\KeywordTok{summarise}\NormalTok{(}\DataTypeTok{game =} \KeywordTok{sum}\NormalTok{(G), }\DataTypeTok{atbat =} \KeywordTok{sum}\NormalTok{(AB), }\DataTypeTok{hit =} \KeywordTok{sum}\NormalTok{(H), }\DataTypeTok{homerun =} \KeywordTok{sum}\NormalTok{(HR), }\DataTypeTok{so =} \KeywordTok{sum}\NormalTok{(SO))}

\CommentTok{# デビューの年をチェックします}
\NormalTok{start =}\StringTok{ }\NormalTok{batters_data_hit %>%}\StringTok{ }
\StringTok{  }\KeywordTok{group_by}\NormalTok{(fullname) %>%}\StringTok{ }
\StringTok{  }\NormalTok{dplyr::}\KeywordTok{summarise}\NormalTok{(}\DataTypeTok{start =} \KeywordTok{min}\NormalTok{(yearID)) }

\CommentTok{# 最近の選手かどうか. 1975以降かどうかで場合分け}
\NormalTok{batters_data_hit =}\StringTok{ }
\StringTok{  }\NormalTok{batters_data_hit %>%}\StringTok{ }\KeywordTok{inner_join}\NormalTok{(start, }\DataTypeTok{by=}\StringTok{"fullname"}\NormalTok{) %>%}\StringTok{ }
\StringTok{  }\KeywordTok{mutate}\NormalTok{(}\DataTypeTok{recent =} \KeywordTok{ifelse}\NormalTok{(start >}\StringTok{ }\DecValTok{1970}\NormalTok{, }\StringTok{"recent"}\NormalTok{, }\StringTok{"old"}\NormalTok{)) }

\CommentTok{# できたデータ}
\KeywordTok{head}\NormalTok{(batters_data_hit) %>%}\StringTok{ }\KeywordTok{xtable}\NormalTok{() %>%}\StringTok{ }\KeywordTok{print}\NormalTok{(}\DataTypeTok{type=}\StringTok{"html"}\NormalTok{)}
\end{Highlighting}
\end{Shaded}

fullname

yearID

game

atbat

hit

homerun

so

start

recent

1

Al Kaline

1953

30

28

7

1

5

1953

old

2

Al Kaline

1954

138

504

139

4

45

1953

old

3

Al Kaline

1955

152

588

200

27

57

1953

old

4

Al Kaline

1956

153

617

194

27

55

1953

old

5

Al Kaline

1957

149

577

170

23

38

1953

old

6

Al Kaline

1958

146

543

170

16

47

1953

old

通算成績の積み上げの様子を見てみます.

\begin{Shaded}
\begin{Highlighting}[]
\CommentTok{# ひたすらcumsumします}
\NormalTok{career_data =}
\StringTok{  }\NormalTok{batters_data_hit %>%}\StringTok{ }\KeywordTok{group_by}\NormalTok{(fullname) %>%}
\StringTok{  }\NormalTok{dplyr::}\KeywordTok{summarise}\NormalTok{(}\DataTypeTok{yearID =} \NormalTok{yearID, }
                   \DataTypeTok{careerHIT=} \KeywordTok{cumsum}\NormalTok{(hit),}
                   \DataTypeTok{careerHR =} \KeywordTok{cumsum}\NormalTok{(homerun), }
                   \DataTypeTok{careerSO =} \KeywordTok{cumsum}\NormalTok{(so), }
                   \DataTypeTok{careerGAME=}\KeywordTok{cumsum}\NormalTok{(game), }
                   \DataTypeTok{recent =} \NormalTok{recent}
                   \NormalTok{)}
\CommentTok{# できたデータ}
\KeywordTok{head}\NormalTok{(career_data) %>%}\StringTok{ }\KeywordTok{xtable}\NormalTok{() %>%}\StringTok{ }\KeywordTok{print}\NormalTok{(}\DataTypeTok{type=}\StringTok{"html"}\NormalTok{)}
\end{Highlighting}
\end{Shaded}

fullname

yearID

careerHIT

careerHR

careerSO

careerGAME

recent

1

Al Kaline

1953

7

1

5

30

old

2

Al Kaline

1954

146

5

50

168

old

3

Al Kaline

1955

346

32

107

320

old

4

Al Kaline

1956

540

59

162

473

old

5

Al Kaline

1957

710

82

200

622

old

6

Al Kaline

1958

880

98

247

768

old

\begin{Shaded}
\begin{Highlighting}[]

\CommentTok{# hPlotで可視化してみる}
\NormalTok{hp2 =}\StringTok{ }\KeywordTok{hPlot}\NormalTok{(}\DataTypeTok{data =} \NormalTok{career_data, }\DataTypeTok{x=}\StringTok{"yearID"}\NormalTok{, }\DataTypeTok{y=}\StringTok{"careerHIT"}\NormalTok{, }\DataTypeTok{group =} \StringTok{"fullname"}\NormalTok{, }\DataTypeTok{type=}\StringTok{"line"}\NormalTok{)                  }
\NormalTok{hp2$}\KeywordTok{chart}\NormalTok{(}\DataTypeTok{forceY =} \StringTok{"#![0]!#"}\NormalTok{)}
\NormalTok{hp2$}\KeywordTok{show}\NormalTok{(}\StringTok{"iframesrc"}\NormalTok{, }\DataTypeTok{cdn =} \OtherTok{TRUE}\NormalTok{)}
\end{Highlighting}
\end{Shaded}

これは通算ヒット数の積み上げの様子です. 1970年代以降に,
レジェンドプレーヤーが固まっているように見えますね.

割と最近の選手の成績だけ見てみると,

\begin{Shaded}
\begin{Highlighting}[]
\NormalTok{data_recent =}\StringTok{ }\NormalTok{career_data %>%}\StringTok{ }\KeywordTok{filter}\NormalTok{(recent ==}\StringTok{ "recent"}\NormalTok{) }
\NormalTok{hp =}\StringTok{ }\KeywordTok{hPlot}\NormalTok{(}\DataTypeTok{data =} \NormalTok{data_recent, }\DataTypeTok{x=}\StringTok{"yearID"}\NormalTok{, }\DataTypeTok{y=}\StringTok{"careerHIT"}\NormalTok{, }\DataTypeTok{group=}\StringTok{"fullname"}\NormalTok{, }\DataTypeTok{type=}\StringTok{"line"}\NormalTok{) }
\NormalTok{hp$}\KeywordTok{show}\NormalTok{(}\StringTok{"iframesrc"}\NormalTok{, }\DataTypeTok{cdn =} \OtherTok{TRUE}\NormalTok{)}
\end{Highlighting}
\end{Shaded}

こんな感じです.

ホームラン数も見たいです.

\begin{Shaded}
\begin{Highlighting}[]
\NormalTok{hp2 =}\StringTok{ }\KeywordTok{hPlot}\NormalTok{(}\DataTypeTok{data =} \NormalTok{career_data, }\DataTypeTok{x=}\StringTok{"yearID"}\NormalTok{, }\DataTypeTok{y=}\StringTok{"careerHR"}\NormalTok{, }\DataTypeTok{group =} \StringTok{"fullname"}\NormalTok{, }\DataTypeTok{type=}\StringTok{"line"}\NormalTok{)                  }
\NormalTok{hp2$}\KeywordTok{chart}\NormalTok{(}\DataTypeTok{forceY =} \StringTok{"#![0]!#"}\NormalTok{)}
\NormalTok{hp2$}\KeywordTok{show}\NormalTok{(}\StringTok{"iframesrc"}\NormalTok{, }\DataTypeTok{cdn =} \OtherTok{TRUE}\NormalTok{)}
\end{Highlighting}
\end{Shaded}

1940年台に何かが起きていることが分かります. 優秀な打者が増えていますし,
ホームランの数も劇的に増えています.

三振の数も見ます.

\begin{Shaded}
\begin{Highlighting}[]
\NormalTok{hp2 =}\StringTok{ }\KeywordTok{hPlot}\NormalTok{(}\DataTypeTok{data =} \NormalTok{career_data, }\DataTypeTok{x=}\StringTok{"yearID"}\NormalTok{, }\DataTypeTok{y=}\StringTok{"careerSO"}\NormalTok{, }\DataTypeTok{group =} \StringTok{"fullname"}\NormalTok{, }\DataTypeTok{type=}\StringTok{"line"}\NormalTok{)                  }
\NormalTok{hp2$}\KeywordTok{chart}\NormalTok{(}\DataTypeTok{forceY =} \StringTok{"#![0]!#"}\NormalTok{)}
\NormalTok{hp2$}\KeywordTok{show}\NormalTok{(}\StringTok{"iframesrc"}\NormalTok{, }\DataTypeTok{cdn =} \OtherTok{TRUE}\NormalTok{)}
\end{Highlighting}
\end{Shaded}

古いデータには欠損があるみたいですね. 最近の選手だけ見ます

\begin{Shaded}
\begin{Highlighting}[]
\NormalTok{hp =}\StringTok{ }\KeywordTok{hPlot}\NormalTok{(}\DataTypeTok{data =} \NormalTok{data_recent, }\DataTypeTok{x=}\StringTok{"yearID"}\NormalTok{, }\DataTypeTok{y=}\StringTok{"careerSO"}\NormalTok{, }\DataTypeTok{group=}\StringTok{"fullname"}\NormalTok{, }\DataTypeTok{type=}\StringTok{"line"}\NormalTok{) }
\NormalTok{hp$}\KeywordTok{show}\NormalTok{(}\StringTok{"iframesrc"}\NormalTok{, }\DataTypeTok{cdn =} \OtherTok{TRUE}\NormalTok{)}
\end{Highlighting}
\end{Shaded}

一人半端ない人がいますね\ldots{}

トニー・グウィン.
http://ja.wikipedia.org/wiki/\%E3\%83\%88\%E3\%83\%8B\%E3\%83\%BC\%E3\%83\%BB\%E3\%82\%B0\%E3\%82\%A6\%E3\%82\%A3\%E3\%83\%B3

``通算の打数/三振比率21.4という数字はバッティングスタイルを比較されるイチロー(約10)と比べても倍以上高い''

ひいい

以上です.

\end{document}
