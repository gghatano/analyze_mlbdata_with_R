\documentclass[]{article}
\usepackage[T1]{fontenc}
\usepackage{lmodern}
\usepackage{amssymb,amsmath}
\usepackage{ifxetex,ifluatex}
\usepackage{fixltx2e} % provides \textsubscript
% use upquote if available, for straight quotes in verbatim environments
\IfFileExists{upquote.sty}{\usepackage{upquote}}{}
\ifnum 0\ifxetex 1\fi\ifluatex 1\fi=0 % if pdftex
  \usepackage[utf8]{inputenc}
\else % if luatex or xelatex
  \ifxetex
    \usepackage{mathspec}
    \usepackage{xltxtra,xunicode}
  \else
    \usepackage{fontspec}
  \fi
  \defaultfontfeatures{Mapping=tex-text,Scale=MatchLowercase}
  \newcommand{\euro}{€}
\fi
% use microtype if available
\IfFileExists{microtype.sty}{\usepackage{microtype}}{}
\usepackage{color}
\usepackage{fancyvrb}
\newcommand{\VerbBar}{|}
\newcommand{\VERB}{\Verb[commandchars=\\\{\}]}
\DefineVerbatimEnvironment{Highlighting}{Verbatim}{commandchars=\\\{\}}
% Add ',fontsize=\small' for more characters per line
\newenvironment{Shaded}{}{}
\newcommand{\KeywordTok}[1]{\textcolor[rgb]{0.00,0.44,0.13}{\textbf{{#1}}}}
\newcommand{\DataTypeTok}[1]{\textcolor[rgb]{0.56,0.13,0.00}{{#1}}}
\newcommand{\DecValTok}[1]{\textcolor[rgb]{0.25,0.63,0.44}{{#1}}}
\newcommand{\BaseNTok}[1]{\textcolor[rgb]{0.25,0.63,0.44}{{#1}}}
\newcommand{\FloatTok}[1]{\textcolor[rgb]{0.25,0.63,0.44}{{#1}}}
\newcommand{\CharTok}[1]{\textcolor[rgb]{0.25,0.44,0.63}{{#1}}}
\newcommand{\StringTok}[1]{\textcolor[rgb]{0.25,0.44,0.63}{{#1}}}
\newcommand{\CommentTok}[1]{\textcolor[rgb]{0.38,0.63,0.69}{\textit{{#1}}}}
\newcommand{\OtherTok}[1]{\textcolor[rgb]{0.00,0.44,0.13}{{#1}}}
\newcommand{\AlertTok}[1]{\textcolor[rgb]{1.00,0.00,0.00}{\textbf{{#1}}}}
\newcommand{\FunctionTok}[1]{\textcolor[rgb]{0.02,0.16,0.49}{{#1}}}
\newcommand{\RegionMarkerTok}[1]{{#1}}
\newcommand{\ErrorTok}[1]{\textcolor[rgb]{1.00,0.00,0.00}{\textbf{{#1}}}}
\newcommand{\NormalTok}[1]{{#1}}
\usepackage{longtable,booktabs}
\ifxetex
  \usepackage[setpagesize=false, % page size defined by xetex
              unicode=false, % unicode breaks when used with xetex
              xetex]{hyperref}
\else
  \usepackage[unicode=true]{hyperref}
\fi
\hypersetup{breaklinks=true,
            bookmarks=true,
            pdfauthor={},
            pdftitle={},
            colorlinks=true,
            citecolor=blue,
            urlcolor=blue,
            linkcolor=magenta,
            pdfborder={0 0 0}}
\urlstyle{same}  % don't use monospace font for urls
\setlength{\parindent}{0pt}
\setlength{\parskip}{6pt plus 2pt minus 1pt}
\setlength{\emergencystretch}{3em}  % prevent overfull lines
\setcounter{secnumdepth}{0}

\author{}
\date{}

\begin{document}

\section{1打席で生成する打点の平均値}\label{ux6253ux5e2dux3067ux751fux6210ux3059ux308bux6253ux70b9ux306eux5e73ux5747ux5024}

\subsection{勝負強さとは}\label{ux52ddux8ca0ux5f37ux3055ux3068ux306f}

勝負強さが知りたいです. 点をとってくれる打者を評価したいです.

勝負強さの指標としては, 得点圏打率があげられると思います.

しかし, 得点圏打率には, 打点が反映されていません. また,
生の打点の数字は, 打順に依存します. 勝負強くない打者でも,
チャンスの場面でたくさん打席が回ってくれば,
勝手に打点が上がっていきます.

新しい指標が必要だと思っています.

\subsection{期待打点}\label{ux671fux5f85ux6253ux70b9}

なので, ランナー状況, アウトカウント別に,
打点の期待値を計算してみました.

たとえば,

「1アウト2, 3塁で迎えた打席では何点生まれることが期待できるか」

などの集計を行った, ということです.

そして, 各バッターが,
各打席で期待打点をどれくらい上回ったか\ldots{}という計算を行うことによって,
得点能力が分かりませんかね?

ちょっとやってみます.

\subsection{集計}\label{ux96c6ux8a08}

データ読み込み. 2013年の全打席結果.

\begin{Shaded}
\begin{Highlighting}[]
\KeywordTok{library}\NormalTok{(data.table)}
\KeywordTok{library}\NormalTok{(dplyr)}
\KeywordTok{library}\NormalTok{(xtable)}

\NormalTok{year =}\StringTok{ }\DecValTok{2013}
\NormalTok{file =}\StringTok{ }\KeywordTok{paste}\NormalTok{(}\StringTok{"../../../../data/all"}\NormalTok{, year, }\StringTok{".csv"}\NormalTok{, }\DataTypeTok{sep=}\StringTok{""}\NormalTok{)}
\NormalTok{dat =}\StringTok{ }\KeywordTok{fread}\NormalTok{(file)}
\end{Highlighting}
\end{Shaded}

Read 99.5\% of 190907 rowsRead 190907 rows and 97 (of 97) columns from
0.076 GB file in 00:00:03

\begin{Shaded}
\begin{Highlighting}[]
\NormalTok{names =}\StringTok{ }\KeywordTok{fread}\NormalTok{(}\StringTok{"../names.csv"}\NormalTok{, }\DataTypeTok{header =} \OtherTok{FALSE}\NormalTok{) %>%}\StringTok{ }\NormalTok{unlist}
\NormalTok{dat %>%}\StringTok{ }\KeywordTok{setnames}\NormalTok{(names)}
\end{Highlighting}
\end{Shaded}

ランナー状況を確認. ``100''なら, ランナー3塁です. 各バッターについて,
アウトカウントとランナー状況別, 打席数と挙げた打点(RBI)を集計.

\begin{Shaded}
\begin{Highlighting}[]
\NormalTok{dat_rbi =}\StringTok{ }
\StringTok{  }\NormalTok{dat %>%}\StringTok{ }
\StringTok{  }\CommentTok{#dplyr::filter(AB_FL == "T") %>% }
\StringTok{  }\KeywordTok{mutate}\NormalTok{(}\DataTypeTok{runner =} \NormalTok{(BASE3_RUN_ID !=}\StringTok{ ""}\NormalTok{)*}\DecValTok{100} \NormalTok{+}\StringTok{ }\NormalTok{(BASE2_RUN_ID !=}\StringTok{ ""}\NormalTok{)*}\DecValTok{10} \NormalTok{+}\StringTok{ }\NormalTok{(BASE1_RUN_ID !=}\StringTok{ ""}\NormalTok{)*}\DecValTok{1}\NormalTok{) %>%}\StringTok{ }
\StringTok{  }\KeywordTok{mutate}\NormalTok{(}\DataTypeTok{runner =} \KeywordTok{as.integer}\NormalTok{(runner)) %>%}
\StringTok{  }\KeywordTok{select}\NormalTok{(BAT_ID, OUTS_CT, runner, RBI_CT) %>%}
\StringTok{  }\KeywordTok{group_by}\NormalTok{(BAT_ID, OUTS_CT, runner) %>%}\StringTok{ }
\StringTok{  }\NormalTok{dplyr::}\KeywordTok{summarise}\NormalTok{(}\DataTypeTok{atbat =} \KeywordTok{n}\NormalTok{(), }\DataTypeTok{RBI =} \KeywordTok{sum}\NormalTok{(RBI_CT)) }

\NormalTok{dat_rbi %>%}\StringTok{ }\NormalTok{head %>%}\StringTok{ }\NormalTok{xtable %>%}\StringTok{ }\KeywordTok{print}\NormalTok{(}\StringTok{"html"}\NormalTok{)}
\end{Highlighting}
\end{Shaded}

\begin{longtable}[c]{@{}llllll@{}}
\toprule\addlinespace
& BAT\_ID & OUTS\_CT & runner & atbat & RBI
\\\addlinespace
\midrule\endhead
1 & abret001 & 0 & 0 & 30 & 0
\\\addlinespace
2 & abret001 & 0 & 1 & 10 & 0
\\\addlinespace
3 & abret001 & 0 & 10 & 1 & 0
\\\addlinespace
4 & abret001 & 0 & 11 & 2 & 0
\\\addlinespace
5 & abret001 & 0 & 100 & 1 & 0
\\\addlinespace
6 & abret001 & 0 & 101 & 1 & 0
\\\addlinespace
\bottomrule
\end{longtable}

まずは全バッターで平均をとります. ランナー状況ごとにあげた打点を集計.

\begin{Shaded}
\begin{Highlighting}[]
\NormalTok{dat_rbi_all =}\StringTok{ }
\StringTok{  }\NormalTok{dat_rbi %>%}
\StringTok{  }\KeywordTok{group_by}\NormalTok{(runner, OUTS_CT) %>%}
\StringTok{  }\NormalTok{dplyr::}\KeywordTok{summarise}\NormalTok{(}\DataTypeTok{atbat =} \KeywordTok{sum}\NormalTok{(atbat), }\DataTypeTok{RBI =} \KeywordTok{sum}\NormalTok{(RBI)) %>%}\StringTok{ }
\StringTok{  }\KeywordTok{mutate}\NormalTok{(}\DataTypeTok{RBI_mean =} \NormalTok{RBI /}\StringTok{ }\NormalTok{atbat) }

\NormalTok{dat_rbi_all %>%}\StringTok{ }\KeywordTok{xtable}\NormalTok{(}\DataTypeTok{digits =} \DecValTok{4}\NormalTok{) %>%}\StringTok{ }\KeywordTok{print}\NormalTok{(}\StringTok{"html"}\NormalTok{)}
\end{Highlighting}
\end{Shaded}

\begin{longtable}[c]{@{}llllll@{}}
\toprule\addlinespace
& runner & OUTS\_CT & atbat & RBI & RBI\_mean
\\\addlinespace
\midrule\endhead
1 & 0 & 0 & 45601 & 1347 & 0.0295
\\\addlinespace
2 & 0 & 1 & 32877 & 831 & 0.0253
\\\addlinespace
3 & 0 & 2 & 26180 & 633 & 0.0242
\\\addlinespace
4 & 1 & 0 & 10996 & 695 & 0.0632
\\\addlinespace
5 & 1 & 1 & 13071 & 831 & 0.0636
\\\addlinespace
6 & 1 & 2 & 13385 & 963 & 0.0719
\\\addlinespace
7 & 10 & 0 & 3357 & 443 & 0.1320
\\\addlinespace
8 & 10 & 1 & 5653 & 832 & 0.1472
\\\addlinespace
9 & 10 & 2 & 7307 & 1223 & 0.1674
\\\addlinespace
10 & 11 & 0 & 2584 & 503 & 0.1947
\\\addlinespace
11 & 11 & 1 & 4609 & 996 & 0.2161
\\\addlinespace
12 & 11 & 2 & 5753 & 1343 & 0.2334
\\\addlinespace
13 & 100 & 0 & 476 & 211 & 0.4433
\\\addlinespace
14 & 100 & 1 & 1833 & 874 & 0.4768
\\\addlinespace
15 & 100 & 2 & 2948 & 613 & 0.2079
\\\addlinespace
16 & 101 & 0 & 977 & 477 & 0.4882
\\\addlinespace
17 & 101 & 1 & 2088 & 1059 & 0.5072
\\\addlinespace
18 & 101 & 2 & 2946 & 786 & 0.2668
\\\addlinespace
19 & 110 & 0 & 601 & 399 & 0.6639
\\\addlinespace
20 & 110 & 1 & 1525 & 847 & 0.5554
\\\addlinespace
21 & 110 & 2 & 1881 & 609 & 0.3238
\\\addlinespace
22 & 111 & 0 & 630 & 446 & 0.7079
\\\addlinespace
23 & 111 & 1 & 1621 & 1242 & 0.7662
\\\addlinespace
24 & 111 & 2 & 2008 & 1068 & 0.5319
\\\addlinespace
\bottomrule
\end{longtable}

なるほど. ためしに満塁だけ注目.

\begin{Shaded}
\begin{Highlighting}[]
\NormalTok{dat_rbi_all %>%}
\StringTok{  }\NormalTok{dplyr::}\KeywordTok{filter}\NormalTok{(runner ==}\StringTok{ }\DecValTok{111}\NormalTok{) %>%}
\StringTok{  }\KeywordTok{xtable}\NormalTok{(}\DataTypeTok{digits =} \DecValTok{4}\NormalTok{) %>%}\StringTok{ }\KeywordTok{print}\NormalTok{(}\StringTok{"html"}\NormalTok{)}
\end{Highlighting}
\end{Shaded}

\begin{longtable}[c]{@{}llllll@{}}
\toprule\addlinespace
& runner & OUTS\_CT & atbat & RBI & RBI\_mean
\\\addlinespace
\midrule\endhead
1 & 111 & 0 & 630 & 446 & 0.7079
\\\addlinespace
2 & 111 & 1 & 1621 & 1242 & 0.7662
\\\addlinespace
3 & 111 & 2 & 2008 & 1068 & 0.5319
\\\addlinespace
\bottomrule
\end{longtable}

0アウト満塁だと, 1打席で0.708点. 1アウト満塁だと, 1打席で0.766点ですか.
0アウト満塁よりも, 1アウト満塁のほうが, 平均打点が高いみたいです.

ちょっと感覚と合いません. 1アウトなら, ゲッツーで終わっちゃいますもんね.
0アウトなら, ゲッツーの間に1点は入ります.

\subsection{本当に差があるといっていいのか?}\label{ux672cux5f53ux306bux5deeux304cux3042ux308bux3068ux3044ux3063ux3066ux3044ux3044ux306eux304b}

本当に1アウト満塁のほうが平均打点が高いのか\ldots{} について検定します.

\begin{Shaded}
\begin{Highlighting}[]
\NormalTok{dat_rbi_atbat =}\StringTok{ }
\StringTok{  }\NormalTok{dat %>%}\StringTok{ }
\StringTok{  }\CommentTok{#dplyr::filter(AB_FL == "T") %>% }
\StringTok{  }\KeywordTok{mutate}\NormalTok{(}\DataTypeTok{runner =} \NormalTok{(BASE1_RUN_ID !=}\StringTok{ ""}\NormalTok{)*}\DecValTok{1} \NormalTok{+}\StringTok{ }\NormalTok{(BASE2_RUN_ID !=}\StringTok{ ""}\NormalTok{)*}\DecValTok{10} \NormalTok{+}\StringTok{ }\NormalTok{(BASE3_RUN_ID !=}\StringTok{ ""}\NormalTok{)*}\DecValTok{100}\NormalTok{) %>%}\StringTok{ }
\StringTok{  }\KeywordTok{mutate}\NormalTok{(}\DataTypeTok{runner =} \KeywordTok{as.integer}\NormalTok{(runner)) %>%}
\StringTok{  }\KeywordTok{select}\NormalTok{(BAT_ID, OUTS_CT, runner, RBI_CT) %>%}
\StringTok{  }\KeywordTok{group_by}\NormalTok{(OUTS_CT, runner, RBI_CT, }\DataTypeTok{add =} \OtherTok{FALSE}\NormalTok{) %>%}\StringTok{ }
\StringTok{  }\NormalTok{dplyr::}\KeywordTok{summarise}\NormalTok{(}\DataTypeTok{atbat =} \KeywordTok{n}\NormalTok{())}

\NormalTok{dat_rbi_atbat_fullbase =}\StringTok{ }
\StringTok{  }\NormalTok{dat_rbi_atbat %>%}\StringTok{ }
\StringTok{  }\NormalTok{dplyr::}\KeywordTok{filter}\NormalTok{(runner ==}\StringTok{ }\DecValTok{111}\NormalTok{, OUTS_CT <}\StringTok{ }\DecValTok{2}\NormalTok{)}
\NormalTok{dat_rbi_atbat_fullbase %>%}\StringTok{ }
\StringTok{  }\NormalTok{xtable %>%}\StringTok{ }\KeywordTok{print}\NormalTok{(}\StringTok{"html"}\NormalTok{)}
\end{Highlighting}
\end{Shaded}

\begin{longtable}[c]{@{}lllll@{}}
\toprule\addlinespace
& OUTS\_CT & runner & RBI\_CT & atbat
\\\addlinespace
\midrule\endhead
1 & 0 & 111 & 0 & 307
\\\addlinespace
2 & 0 & 111 & 1 & 232
\\\addlinespace
3 & 0 & 111 & 2 & 72
\\\addlinespace
4 & 0 & 111 & 3 & 6
\\\addlinespace
5 & 0 & 111 & 4 & 13
\\\addlinespace
6 & 1 & 111 & 0 & 734
\\\addlinespace
7 & 1 & 111 & 1 & 647
\\\addlinespace
8 & 1 & 111 & 2 & 167
\\\addlinespace
9 & 1 & 111 & 3 & 31
\\\addlinespace
10 & 1 & 111 & 4 & 42
\\\addlinespace
\bottomrule
\end{longtable}

0アウト, 1アウトで得られた打点ベクトルを作って,
wilcox.testをかけてみます. 平均の差があるかどうか.

\begin{Shaded}
\begin{Highlighting}[]
\NormalTok{noout_fullbase =}\StringTok{ }
\StringTok{  }\NormalTok{dat_rbi_atbat_fullbase %>%}
\StringTok{  }\NormalTok{dplyr::}\KeywordTok{filter}\NormalTok{(OUTS_CT ==}\StringTok{ }\DecValTok{0}\NormalTok{) %>%}\StringTok{ }
\StringTok{  }\KeywordTok{do}\NormalTok{(}\DataTypeTok{vec =} \KeywordTok{rep}\NormalTok{(RBI_CT, atbat)) %>%}\StringTok{ }
\StringTok{  }\NormalTok{dplyr::}\KeywordTok{summarise}\NormalTok{(}\DataTypeTok{vec =} \NormalTok{vec) %>%}\StringTok{ }
\StringTok{  }\NormalTok{unlist }
\NormalTok{oneout_fullbase =}\StringTok{ }
\StringTok{  }\NormalTok{dat_rbi_atbat_fullbase %>%}
\StringTok{  }\NormalTok{dplyr::}\KeywordTok{filter}\NormalTok{(OUTS_CT ==}\StringTok{ }\DecValTok{1}\NormalTok{) %>%}\StringTok{ }
\StringTok{  }\KeywordTok{do}\NormalTok{(}\DataTypeTok{vec =} \KeywordTok{rep}\NormalTok{(RBI_CT, atbat)) %>%}\StringTok{ }
\StringTok{  }\NormalTok{dplyr::}\KeywordTok{summarise}\NormalTok{(}\DataTypeTok{vec =} \NormalTok{vec) %>%}\StringTok{ }
\StringTok{  }\NormalTok{unlist}

\NormalTok{noout_fullbase %>%}\StringTok{ }\NormalTok{table %>%}\StringTok{ }\NormalTok{xtable %>%}\StringTok{ }\KeywordTok{print}\NormalTok{(}\StringTok{"html"}\NormalTok{)}
\end{Highlighting}
\end{Shaded}

\begin{longtable}[c]{@{}ll@{}}
\toprule\addlinespace
& noout\_fullbase
\\\addlinespace
\midrule\endhead
0 & 307
\\\addlinespace
1 & 232
\\\addlinespace
2 & 72
\\\addlinespace
3 & 6
\\\addlinespace
4 & 13
\\\addlinespace
\bottomrule
\end{longtable}

\begin{Shaded}
\begin{Highlighting}[]
\NormalTok{oneout_fullbase %>%}\StringTok{ }\NormalTok{table %>%}\StringTok{ }\NormalTok{xtable %>%}\StringTok{ }\KeywordTok{print}\NormalTok{(}\StringTok{"html"}\NormalTok{)}
\end{Highlighting}
\end{Shaded}

\begin{longtable}[c]{@{}ll@{}}
\toprule\addlinespace
& oneout\_fullbase
\\\addlinespace
\midrule\endhead
0 & 734
\\\addlinespace
1 & 647
\\\addlinespace
2 & 167
\\\addlinespace
3 & 31
\\\addlinespace
4 & 42
\\\addlinespace
\bottomrule
\end{longtable}

ここのコードがダサいですね\ldots{}

次. ウィルコックスの順位和検定をかけます.

\begin{Shaded}
\begin{Highlighting}[]
\KeywordTok{wilcox.test}\NormalTok{(noout_fullbase, oneout_fullbase, }\DataTypeTok{conf.int =} \OtherTok{TRUE}\NormalTok{)}
\end{Highlighting}
\end{Shaded}

\begin{verbatim}
## 
##  Wilcoxon rank sum test with continuity correction
## 
## data:  noout_fullbase and oneout_fullbase
## W = 493634, p-value = 0.1809
## alternative hypothesis: true location shift is not equal to 0
## 95 percent confidence interval:
##  -8.866e-06  6.953e-06
## sample estimates:
## difference in location 
##             -3.235e-06
\end{verbatim}

0アウト満塁と1アウト満塁. 平均に差がない,
という帰無仮説を棄却できませんでした(p-valueは0.18).

ありがとうございました.

\subsection{今後の予定}\label{ux4ecaux5f8cux306eux4e88ux5b9a}

打者ごとに期待打点をどの程度上回ったか,についてはまだ計算していません.
次にやります.

また, 今は2013年のデータだけを使っていますが,
他の年のデータも合併したものでチェックしてもいい気がします.

\end{document}
